\documentclass{article}
\usepackage[utf8]{inputenc}
\usepackage[spanish]{babel}
\usepackage{listings}
\usepackage{graphicx}
\graphicspath{ {images/} }
\usepackage{cite}

\begin{document}

\begin{titlepage}
    \begin{center}
        \vspace*{1cm}
            
        \Huge
        \textbf{Calistelia}
            
        \vspace{0.5cm}
        \LARGE
        Informatica 2
            
        \vspace{1.5cm}
            
        \textbf{Juan Camilo Arboleda Arboleda}
        
        \vspace{0.5cm}
            
        \textbf{C.C. 1000850460}
            
        \vfill
            
        \vspace{0.8cm}
            
        \Large
        Despartamento de Ingeniería Electrónica y Telecomunicaciones\\
        Universidad de Antioquia\\
        Medellín\\
        Marzo de 2021
            
    \end{center}
\end{titlepage}

\tableofcontents
\newpage
\section{Objetos necesarios}\label{intro}
En el siguiente experimento necesitaremos una hoja nueva de un bloc de cualquier tamaño, dos tarjetas del mismo tamaño y si es posible del mismo grosor, una superficie plana horizontal donde apoyar todos los objetos como una mesa o escritorio

\section{Situación inicial}\label{intro}
La situacion inicial a la que nos vamos a enfrentar es poner las tarjetas sobre la mesa y luego poner la hoja sobre las tarjetas 


\section{Procedimiento} \label{contenido}

\subsection{Aclaración}
Para este experimento debemos tener en cuenta que la hoja tendra largo y ancho, siendo el largo el de mayor medida llamado también eje (y), en la mitad del eje(y) se ubicara el eje(x) de forma paralela al ancho de la pagina 

Tendremos en cuenta que los dedos de la mano se enumeraran del 1 al 5 donde el pulgar sera el numero 1, indice el 2, medio el 3, anular el 4 y el meñique el 5.

Por ultimo aclarar el largo y ancho de las tarjertas donde el largo será el lado con más medida

\subsection{Paso 1}
Vamos a tomar la hoja con la mano derecha y la moveremos hacia la derecha de tal manera que quede a unos 10 cm de las tarjetas

\subsection{Paso 2}
Se toma las tarjetas con la mano derecha si es necesario dejandolas  de forma paralelas y unidas, luego de organizarlas se ubicara el dedo 1 y 3 de la mano derecha en las esquinas de cualquier lado ancho dejando la mayor parte de las tarjetas hacia la mesa en la que se realiza el experimento

\subsection{Paso 3}
Con la izquierda se corre la hoja de papel de tal manera que vuelva a la posición donde se econtraba al inicio 

\subsection{Paso 4}
Baja las tarjetas de tal manera que queden perpendicular al eje (x) apoyandolas sobre la hoja de papel

\subsection{Paso 5}
Se ubica el dedo 2 sobre las tarjetas apoyandola en la parte ancha que no esta sobre la hoja y se hace presíon suficiente para que no se caigan las tarjetas, se sueltan los dedos 1 y 3

\subsection{Paso 6}
con los dedos 1, 3 y 4 se comienza a separar la parte ancha de las tarjetas que se apoya sobre la hoja sin dejar que se separen las tarjetas se levanten de la hoja y que la parte superior de las tarjetas sostenido por el dedo 2 unido.

\subsection{Paso 7}
La separación de las tarjetas se hacen hasta que sean capaces de sostenerse por si mismas, cuando esto sea posible se levanta el dedo 2 y con esto se finaliza el experimento

\end{document}